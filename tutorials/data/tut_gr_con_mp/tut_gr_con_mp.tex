\documentclass[justified]{tufte-handout}
\usepackage{../braph2_tut}
%\geometry{showframe} % display margins for debugging page layout

\title{Group of Subjects with Connectivity Multiplex Data}

\begin{document}

\maketitle

\begin{abstract}
\noindent
For \emph{connectivity multiplex data}, a connectivity matrix per subject is already available for different layers and can be directly imported into the relative analysis pipeline. For example, the connectivity matrix could correspond to white matter tracts obtained from dMRI or pre-calculated coactivations maps obtained from fMRI data.
This Tutorial explains how to prepare and work with this kind of data.
\end{abstract}

%! FIG1 !%

\tableofcontents

\fig{figure}
	{fig:01}
	{\includegraphics{fig01.jpg}}
	{GUI for a group of subjects with connectivity multiplex data}
	{
	Full graphical user interface to upload a group of subjects with connectivity multiplex data in BRAPH~2. 
	}

\clearpage
\section{Generation of Example Data}

If you do not have the \fn{Example data CON\_MP XLS} folder inside \fn{connectivity multiplex}, then you can generate it by running the commands in \Coderef{cd:exampledata}.

\begin{lstlisting}[
	label=cd:exampledata,
	caption={
		{\bf Code to generate the example data folder.}
		This code can be used in the MatLab command line to generate the \fn{Example data CON\_MP XLS} folder to the \fn{connectivity multiplex} pipeline folder.
	}
]
create_data_CON_MP_XLS() ¥\circled{1}\circlednote{1}{ generates the example connectivity multiplex XLS data folder.}¥
create_data_CON_MP_TXT() ¥\circled{2}\circlednote{2}{ generates the example connectivity multiplex TXT data folder.}¥
\end{lstlisting}

\section{Open the GUI}

In most analyses, the group GUI is the second step after you have selected a brain atlas. You can open it by typing \code{braph2} in MatLab's terminal, which allows you to select a pipeline containing the steps required to perform your analysis and upload a brain atlas. After these steps have been completed you can upload your group's data directly (\Figref{fig:02}c-f) after clicking ``Load Group''.

\fig{figure}
	{fig:02}
	{
	\includegraphics{fig02.jpg}
	}
	{Upload the data of a group of subjects}
	{
	Steps to upload a group of subjects with connectivity multiplex data using the GUI and an example dataset: 
	{\bf a} Open the group GUI.
	{\bf b} Import a folder that contains one file per subject and layer with the connectivity matrix in XLS or TXT format (see below for details on their format).
	To upload the test connectivity multiplex data:
	{\bf c}-{\bf f} navigate to the BRAPH~2 folder \fn{pipelines}, {\bf d} \fn{connectivity multiplex},  {\bf e} \fn{Example data CON\_MP XLS}, and {\bf f} select the folder containing the connectivity matrices of one group \fn{CON\_MP\_Group\_1\_XLS}.
	}

%! FIG2 !%

\begin{tcolorbox}[
	title=GUI launch from command line
]
You can also open the GUI and upload the brain connectivity multiplex data using the command line (i.e., without opening an analysis pipeline) by typing the commands in \Coderef{cd:launch}. In this case, you can upload the data as shown in \Figref{fig:02}a-f.

\begin{lstlisting}[
	label=cd:launch,
	caption={
		{\bf Code to launch the GUI to upload a group of subjects with connectivity multiplex data.}
		This code can be used in the MatLab command line to launch the GUI to upload a group of subjects with connectivity multiplex data without having to open a pipeline.
	}
]
gr = Group('SUB_CLASS', 'SubjectCON_MP');

gui = GUIElement('PE', gr);
gui.get('DRAW')
gui.get('SHOW')
\end{lstlisting}
\end{tcolorbox}

\section{Visualize the Group Data}

After completing the steps described in \Figref{fig:02}, you can see the data (\Figref{fig:03}a), and change the Group ID, name, and notes (\Figref{fig:03}b). 

\fig{figure}
	{fig:03}
	{
	\includegraphics{fig03.jpg}
	}
	{Edit the group metadata}
	{ 
	{\bf a} The GUI of the group's connectivity multiplex data. 
	{\bf b} The information you see on this GUI that can be changed. In this example, we have edited the ID, name, and notes of the group but can also change the subject's specific information.
	}

%! FIG3 !%

\section{Visualize Each Subject's Data}

Finally, you can open each subject's connectivity multiplex data by selecting the subject, right click, and select ``Open selection'' (\Figref{fig:04}a), which shows the matrix values from layer 1 (\Figref{fig:04}b). Here, you can also change the subject's metadata (ID, label, notes), its variables of interest, and the values of its connectivity multiplex data.

\fig{figure}
	{fig:04}
	{\includegraphics{fig04.jpg}
	}
	{Edit the individual subject data}
	{
	{\bf a}  Each subject's connectivity multiplex data can be opened by selecting the subject, right click, and select ''Open selection''. 
	{\bf b} In this subject GUI, it is possible to view and edit the metadata of the subject (ID, label, notes), its variables of interest (in this case, age and sex), and the connectivity multiplex data. 
	}

%! FIG4 !%

\clearpage
\section{Preparation of the Data to Be Imported}

To be able to import connectivity multiplex data into BRAPH~2, you create a folder with the name of your group, and within this group folder, you need to include the connectivity matrices for each subject and for each layer in a single file in excel or text format. Below you can see how your group directory should look like as well as an example of a brain connectivity matrix.

\fig{figure*}
	{fig:05}
	{
	[h!]
	\includegraphics{fig05.jpg}
	}
	{Data preparation}
	{
	The data should be organised in the following format:
	{\bf a} The connectivity matrices from each subject and each layer should be included in one folder (for example, \fn{CON\_MP\_group\_1\_XLS}). 
	{\bf b} Each matrix should contain the connectivity values between each pair of brain regions denoted by the rows and columns. In this example, the (simulated) values in the matrix correspond to the fractional anisotropy (white matter integrity) of anatomical connections derived from diffusion weighted imaging.
	}
	
%! FIG5 !%

\section{Adding Covariates}

\fig{figure*}
	{fig:06}
	{
	[h!]
	\includegraphics{fig06.jpg}
	}
	{Edit the Covariates}
	{
	Information that can be changed in the Covariates file: 
	{\bf a} The names of the variables of interest (vois).
	{\bf b} In case the vois are categorical, you can state which categories they have.
	}
	
It is very common to have \emph{variables of interest} (i.e., \emph{covariates} and \emph{correlates}) in an analysis. In BRAPH~2, these variables of interest should be included in a separate excel file placed just outside the group's folder and with the same name as the folder followed by \fn{.vois} (\Figref{fig:06}a). This file should have a specific format (\Figref{fig:06}b):
\begin{itemize}

\item {\bf Subject IDs (column A).}
Column A should contain the subject IDs starting from row 3.

\item {\bf Variables of interest (column B and subsequent columns).}
Column B (and subsequent columns) should contain the variables of interest (one per column). 
In this example we have ``Age'' and ``Sex'', as in the example file, as well as the additional ``Education''.
In each column, row 1 should contain the name of the variable of interest, row 2 should contain the categories separated by a return (only for categorical variables of interest, like ``Sex'' and ``Education''), and the subsequent rows the values of the variable of interest for each subject.

\end{itemize}

%! FIG6 !%

\end{document}